% Options for packages loaded elsewhere
\PassOptionsToPackage{unicode}{hyperref}
\PassOptionsToPackage{hyphens}{url}
%
\documentclass[
]{article}
\usepackage{amsmath,amssymb}
\usepackage{iftex}
\ifPDFTeX
  \usepackage[T1]{fontenc}
  \usepackage[utf8]{inputenc}
  \usepackage{textcomp} % provide euro and other symbols
\else % if luatex or xetex
  \usepackage{unicode-math} % this also loads fontspec
  \defaultfontfeatures{Scale=MatchLowercase}
  \defaultfontfeatures[\rmfamily]{Ligatures=TeX,Scale=1}
\fi
\usepackage{lmodern}
\ifPDFTeX\else
  % xetex/luatex font selection
\fi
% Use upquote if available, for straight quotes in verbatim environments
\IfFileExists{upquote.sty}{\usepackage{upquote}}{}
\IfFileExists{microtype.sty}{% use microtype if available
  \usepackage[]{microtype}
  \UseMicrotypeSet[protrusion]{basicmath} % disable protrusion for tt fonts
}{}
\makeatletter
\@ifundefined{KOMAClassName}{% if non-KOMA class
  \IfFileExists{parskip.sty}{%
    \usepackage{parskip}
  }{% else
    \setlength{\parindent}{0pt}
    \setlength{\parskip}{6pt plus 2pt minus 1pt}}
}{% if KOMA class
  \KOMAoptions{parskip=half}}
\makeatother
\usepackage{xcolor}
\usepackage[margin=1in]{geometry}
\usepackage{color}
\usepackage{fancyvrb}
\newcommand{\VerbBar}{|}
\newcommand{\VERB}{\Verb[commandchars=\\\{\}]}
\DefineVerbatimEnvironment{Highlighting}{Verbatim}{commandchars=\\\{\}}
% Add ',fontsize=\small' for more characters per line
\usepackage{framed}
\definecolor{shadecolor}{RGB}{248,248,248}
\newenvironment{Shaded}{\begin{snugshade}}{\end{snugshade}}
\newcommand{\AlertTok}[1]{\textcolor[rgb]{0.94,0.16,0.16}{#1}}
\newcommand{\AnnotationTok}[1]{\textcolor[rgb]{0.56,0.35,0.01}{\textbf{\textit{#1}}}}
\newcommand{\AttributeTok}[1]{\textcolor[rgb]{0.13,0.29,0.53}{#1}}
\newcommand{\BaseNTok}[1]{\textcolor[rgb]{0.00,0.00,0.81}{#1}}
\newcommand{\BuiltInTok}[1]{#1}
\newcommand{\CharTok}[1]{\textcolor[rgb]{0.31,0.60,0.02}{#1}}
\newcommand{\CommentTok}[1]{\textcolor[rgb]{0.56,0.35,0.01}{\textit{#1}}}
\newcommand{\CommentVarTok}[1]{\textcolor[rgb]{0.56,0.35,0.01}{\textbf{\textit{#1}}}}
\newcommand{\ConstantTok}[1]{\textcolor[rgb]{0.56,0.35,0.01}{#1}}
\newcommand{\ControlFlowTok}[1]{\textcolor[rgb]{0.13,0.29,0.53}{\textbf{#1}}}
\newcommand{\DataTypeTok}[1]{\textcolor[rgb]{0.13,0.29,0.53}{#1}}
\newcommand{\DecValTok}[1]{\textcolor[rgb]{0.00,0.00,0.81}{#1}}
\newcommand{\DocumentationTok}[1]{\textcolor[rgb]{0.56,0.35,0.01}{\textbf{\textit{#1}}}}
\newcommand{\ErrorTok}[1]{\textcolor[rgb]{0.64,0.00,0.00}{\textbf{#1}}}
\newcommand{\ExtensionTok}[1]{#1}
\newcommand{\FloatTok}[1]{\textcolor[rgb]{0.00,0.00,0.81}{#1}}
\newcommand{\FunctionTok}[1]{\textcolor[rgb]{0.13,0.29,0.53}{\textbf{#1}}}
\newcommand{\ImportTok}[1]{#1}
\newcommand{\InformationTok}[1]{\textcolor[rgb]{0.56,0.35,0.01}{\textbf{\textit{#1}}}}
\newcommand{\KeywordTok}[1]{\textcolor[rgb]{0.13,0.29,0.53}{\textbf{#1}}}
\newcommand{\NormalTok}[1]{#1}
\newcommand{\OperatorTok}[1]{\textcolor[rgb]{0.81,0.36,0.00}{\textbf{#1}}}
\newcommand{\OtherTok}[1]{\textcolor[rgb]{0.56,0.35,0.01}{#1}}
\newcommand{\PreprocessorTok}[1]{\textcolor[rgb]{0.56,0.35,0.01}{\textit{#1}}}
\newcommand{\RegionMarkerTok}[1]{#1}
\newcommand{\SpecialCharTok}[1]{\textcolor[rgb]{0.81,0.36,0.00}{\textbf{#1}}}
\newcommand{\SpecialStringTok}[1]{\textcolor[rgb]{0.31,0.60,0.02}{#1}}
\newcommand{\StringTok}[1]{\textcolor[rgb]{0.31,0.60,0.02}{#1}}
\newcommand{\VariableTok}[1]{\textcolor[rgb]{0.00,0.00,0.00}{#1}}
\newcommand{\VerbatimStringTok}[1]{\textcolor[rgb]{0.31,0.60,0.02}{#1}}
\newcommand{\WarningTok}[1]{\textcolor[rgb]{0.56,0.35,0.01}{\textbf{\textit{#1}}}}
\usepackage{graphicx}
\makeatletter
\def\maxwidth{\ifdim\Gin@nat@width>\linewidth\linewidth\else\Gin@nat@width\fi}
\def\maxheight{\ifdim\Gin@nat@height>\textheight\textheight\else\Gin@nat@height\fi}
\makeatother
% Scale images if necessary, so that they will not overflow the page
% margins by default, and it is still possible to overwrite the defaults
% using explicit options in \includegraphics[width, height, ...]{}
\setkeys{Gin}{width=\maxwidth,height=\maxheight,keepaspectratio}
% Set default figure placement to htbp
\makeatletter
\def\fps@figure{htbp}
\makeatother
\setlength{\emergencystretch}{3em} % prevent overfull lines
\providecommand{\tightlist}{%
  \setlength{\itemsep}{0pt}\setlength{\parskip}{0pt}}
\setcounter{secnumdepth}{-\maxdimen} % remove section numbering
\ifLuaTeX
  \usepackage{selnolig}  % disable illegal ligatures
\fi
\IfFileExists{bookmark.sty}{\usepackage{bookmark}}{\usepackage{hyperref}}
\IfFileExists{xurl.sty}{\usepackage{xurl}}{} % add URL line breaks if available
\urlstyle{same}
\hypersetup{
  pdftitle={Erdbeben},
  hidelinks,
  pdfcreator={LaTeX via pandoc}}

\title{Erdbeben}
\author{}
\date{\vspace{-2.5em}}

\begin{document}
\maketitle

\hypertarget{ausgangslage}{%
\section{Ausgangslage}\label{ausgangslage}}

Für eine Versicherung soll ein News-Feed erstellt werden, welcher
kontinuierlich die Erdbebendaten von der Internetseite herunter lädt,
über die Pipeline aufbereitet und in einem Dashboard anzeigt.

Die Daten werden von der Website
\url{https://earthquake.usgs.gov/earthquakes/feed/v1.0/geojson.php}
geladen

\begin{Shaded}
\begin{Highlighting}[]
\CommentTok{\# function to check if package is present}
\NormalTok{install\_if\_not\_present }\OtherTok{\textless{}{-}} \ControlFlowTok{function}\NormalTok{(pkg\_name)\{}
  \ControlFlowTok{if}\NormalTok{(}\SpecialCharTok{!}\FunctionTok{requireNamespace}\NormalTok{(pkg\_name, }\AttributeTok{quietly =} \ConstantTok{TRUE}\NormalTok{))\{}
    \FunctionTok{install.packages}\NormalTok{(pkg\_name)}
\NormalTok{  \}}
\NormalTok{\}}

\FunctionTok{install\_if\_not\_present}\NormalTok{(}\StringTok{"tidyverse"}\NormalTok{)}
\FunctionTok{install\_if\_not\_present}\NormalTok{(}\StringTok{"geojsonio"}\NormalTok{)}
\end{Highlighting}
\end{Shaded}

\begin{verbatim}
## Registered S3 method overwritten by 'geojsonsf':
##   method        from   
##   print.geojson geojson
\end{verbatim}

\begin{Shaded}
\begin{Highlighting}[]
\FunctionTok{install\_if\_not\_present}\NormalTok{(}\StringTok{"geojsonR"}\NormalTok{)}
\FunctionTok{install\_if\_not\_present}\NormalTok{(}\StringTok{"sf"}\NormalTok{)}
\FunctionTok{install\_if\_not\_present}\NormalTok{(}\StringTok{"tidyjson"}\NormalTok{)}
\FunctionTok{install\_if\_not\_present}\NormalTok{(}\StringTok{"jsonlite"}\NormalTok{)}
\FunctionTok{install\_if\_not\_present}\NormalTok{(}\StringTok{"lubridate"}\NormalTok{)}
\FunctionTok{install\_if\_not\_present}\NormalTok{(}\StringTok{"ggplot2"}\NormalTok{)}
\FunctionTok{install\_if\_not\_present}\NormalTok{(}\StringTok{"leaflet"}\NormalTok{)}

\FunctionTok{library}\NormalTok{(tidyverse)}
\end{Highlighting}
\end{Shaded}

\begin{verbatim}
## -- Attaching core tidyverse packages ------------------------ tidyverse 2.0.0 --
## v dplyr     1.1.3     v readr     2.1.4
## v forcats   1.0.0     v stringr   1.5.0
## v ggplot2   3.4.4     v tibble    3.2.1
## v lubridate 1.9.3     v tidyr     1.3.0
## v purrr     1.0.2
\end{verbatim}

\begin{verbatim}
## -- Conflicts ------------------------------------------ tidyverse_conflicts() --
## x dplyr::filter() masks stats::filter()
## x dplyr::lag()    masks stats::lag()
## i Use the conflicted package (<http://conflicted.r-lib.org/>) to force all conflicts to become errors
\end{verbatim}

\begin{Shaded}
\begin{Highlighting}[]
\FunctionTok{library}\NormalTok{(geojsonio)}
\end{Highlighting}
\end{Shaded}

\begin{verbatim}
## 
## Attache Paket: 'geojsonio'
## 
## Das folgende Objekt ist maskiert 'package:base':
## 
##     pretty
\end{verbatim}

\begin{Shaded}
\begin{Highlighting}[]
\FunctionTok{library}\NormalTok{(geojsonR)}
\FunctionTok{library}\NormalTok{(sf)}
\end{Highlighting}
\end{Shaded}

\begin{verbatim}
## Linking to GEOS 3.11.2, GDAL 3.6.2, PROJ 9.2.0; sf_use_s2() is TRUE
\end{verbatim}

\begin{Shaded}
\begin{Highlighting}[]
\FunctionTok{library}\NormalTok{(tidyjson)}
\end{Highlighting}
\end{Shaded}

\begin{verbatim}
## 
## Attache Paket: 'tidyjson'
## 
## Das folgende Objekt ist maskiert 'package:stats':
## 
##     filter
\end{verbatim}

\begin{Shaded}
\begin{Highlighting}[]
\FunctionTok{library}\NormalTok{(jsonlite)}
\end{Highlighting}
\end{Shaded}

\begin{verbatim}
## 
## Attache Paket: 'jsonlite'
## 
## Das folgende Objekt ist maskiert 'package:tidyjson':
## 
##     read_json
## 
## Das folgende Objekt ist maskiert 'package:purrr':
## 
##     flatten
\end{verbatim}

\begin{Shaded}
\begin{Highlighting}[]
\FunctionTok{library}\NormalTok{(lubridate)}
\FunctionTok{library}\NormalTok{(ggplot2)}
\FunctionTok{library}\NormalTok{(leaflet)}
\end{Highlighting}
\end{Shaded}

\hypertarget{aufgabenstellung}{%
\section{Aufgabenstellung}\label{aufgabenstellung}}

\hypertarget{daten-einlesen}{%
\subsection{Daten einlesen}\label{daten-einlesen}}

Datei vom Internet mit den stündlichen Daten laden

\begin{Shaded}
\begin{Highlighting}[]
\NormalTok{past\_hours\_dump }\OtherTok{\textless{}{-}} \FunctionTok{Dump\_From\_GeoJson}\NormalTok{(}\StringTok{"https://earthquake.usgs.gov/earthquakes/feed/v1.0/summary/all\_hour.geojson"}\NormalTok{)}
\FunctionTok{cat}\NormalTok{(past\_hours\_dump)}
\end{Highlighting}
\end{Shaded}

\begin{verbatim}
## {"type":"FeatureCollection","metadata":{"generated":1698512800000,"url":"https://earthquake.usgs.gov/earthquakes/feed/v1.0/summary/all_hour.geojson","title":"USGS All Earthquakes, Past Hour","status":200,"api":"1.10.3","count":5},"features":[{"type":"Feature","properties":{"mag":1.91999996,"place":"5 km SSE of Pāhala, Hawaii","time":1698511592420,"updated":1698511781070,"tz":null,"url":"https://earthquake.usgs.gov/earthquakes/eventpage/hv73629607","detail":"https://earthquake.usgs.gov/earthquakes/feed/v1.0/detail/hv73629607.geojson","felt":null,"cdi":null,"mmi":null,"alert":null,"status":"automatic","tsunami":0,"sig":57,"net":"hv","code":"73629607","ids":",hv73629607,","sources":",hv,","types":",origin,phase-data,","nst":34,"dmin":null,"rms":0.150000006,"gap":172,"magType":"md","type":"earthquake","title":"M 1.9 - 5 km SSE of Pāhala, Hawaii"},"geometry":{"type":"Point","coordinates":[-155.462493896484,19.1601657867432,33.75]},"id":"hv73629607"},
## {"type":"Feature","properties":{"mag":1.9,"place":"Southeastern Alaska","time":1698511176767,"updated":1698511296983,"tz":null,"url":"https://earthquake.usgs.gov/earthquakes/eventpage/ak023du2iu30","detail":"https://earthquake.usgs.gov/earthquakes/feed/v1.0/detail/ak023du2iu30.geojson","felt":null,"cdi":null,"mmi":null,"alert":null,"status":"automatic","tsunami":0,"sig":56,"net":"ak","code":"023du2iu30","ids":",ak023du2iu30,","sources":",ak,","types":",origin,phase-data,","nst":null,"dmin":null,"rms":0.87,"gap":null,"magType":"ml","type":"earthquake","title":"M 1.9 - Southeastern Alaska"},"geometry":{"type":"Point","coordinates":[-136.5545,58.9818,0]},"id":"ak023du2iu30"},
## {"type":"Feature","properties":{"mag":0.97,"place":"4 km NNW of Lake Henshaw, CA","time":1698510873480,"updated":1698511091287,"tz":null,"url":"https://earthquake.usgs.gov/earthquakes/eventpage/ci40590336","detail":"https://earthquake.usgs.gov/earthquakes/feed/v1.0/detail/ci40590336.geojson","felt":null,"cdi":null,"mmi":null,"alert":null,"status":"automatic","tsunami":0,"sig":14,"net":"ci","code":"40590336","ids":",ci40590336,","sources":",ci,","types":",nearby-cities,origin,phase-data,scitech-link,","nst":28,"dmin":0.0384,"rms":0.21,"gap":65,"magType":"ml","type":"earthquake","title":"M 1.0 - 4 km NNW of Lake Henshaw, CA"},"geometry":{"type":"Point","coordinates":[-116.7816667,33.2693333,11.68]},"id":"ci40590336"},
## {"type":"Feature","properties":{"mag":2.2,"place":"Southern Alaska","time":1698510650275,"updated":1698510759235,"tz":null,"url":"https://earthquake.usgs.gov/earthquakes/eventpage/ak023du2gxob","detail":"https://earthquake.usgs.gov/earthquakes/feed/v1.0/detail/ak023du2gxob.geojson","felt":null,"cdi":null,"mmi":null,"alert":null,"status":"automatic","tsunami":0,"sig":74,"net":"ak","code":"023du2gxob","ids":",ak023du2gxob,","sources":",ak,","types":",origin,phase-data,","nst":null,"dmin":null,"rms":0.72,"gap":null,"magType":"ml","type":"earthquake","title":"M 2.2 - Southern Alaska"},"geometry":{"type":"Point","coordinates":[-146.8878,61.3049,17.6]},"id":"ak023du2gxob"},
## {"type":"Feature","properties":{"mag":1.2,"place":"Central Alaska","time":1698509336660,"updated":1698509475900,"tz":null,"url":"https://earthquake.usgs.gov/earthquakes/eventpage/ak023du2c8ey","detail":"https://earthquake.usgs.gov/earthquakes/feed/v1.0/detail/ak023du2c8ey.geojson","felt":null,"cdi":null,"mmi":null,"alert":null,"status":"automatic","tsunami":0,"sig":22,"net":"ak","code":"023du2c8ey","ids":",ak023du2c8ey,","sources":",ak,","types":",origin,phase-data,","nst":null,"dmin":null,"rms":0.72,"gap":null,"magType":"ml","type":"earthquake","title":"M 1.2 - Central Alaska"},"geometry":{"type":"Point","coordinates":[-144.0529,62.8949,3.6]},"id":"ak023du2c8ey"}],"bbox":[-155.46249389648,19.160165786743,0,-116.7816667,62.8949,33.75]}
\end{verbatim}

Zuerst wird der Datensatz mit den Erdbeben der vergangenen Stunde
geladen. Somit kann ein Überblick über die Datenstruktur und die
Datentypen gewonnen werden. Für die weitere Verarbeitung wird die Datei
in ein JSON Objekt eingelesen.

\begin{Shaded}
\begin{Highlighting}[]
\NormalTok{past\_hours\_js }\OtherTok{\textless{}{-}} \FunctionTok{fromJSON}\NormalTok{(past\_hours\_dump)}
\end{Highlighting}
\end{Shaded}

Das JSON-Objekt zeigt eine verschachtelte Struktur mit verschiedenen
Levels, welche einzelne Werte sowie Data Frames enthält. Die
verschiedenen Levels können mittels \$-Zeichen angesprochen und so auch
extrahiert werden.

\hypertarget{daten-aufbereiten}{%
\subsection{Daten aufbereiten}\label{daten-aufbereiten}}

Mit dem JSON Objekt werden Metadaten mitgeliefert, die in die Variable
``metadata'' gespeichert werden.

\begin{Shaded}
\begin{Highlighting}[]
\NormalTok{metadata }\OtherTok{\textless{}{-}}\NormalTok{ past\_hours\_js}\SpecialCharTok{$}\NormalTok{metadata}
\NormalTok{metadata}
\end{Highlighting}
\end{Shaded}

\begin{verbatim}
## $generated
## [1] 1.698513e+12
## 
## $url
## [1] "https://earthquake.usgs.gov/earthquakes/feed/v1.0/summary/all_hour.geojson"
## 
## $title
## [1] "USGS All Earthquakes, Past Hour"
## 
## $status
## [1] 200
## 
## $api
## [1] "1.10.3"
## 
## $count
## [1] 5
\end{verbatim}

Beim Eintrag ``generated'' sieht man, dass das Format nicht dem
gewohnten Format eines Datums entspricht. Dies muss entsprechend
angepasst werden, was zu einem späteren Zeitpunkt gemacht wird.

Das JSON-Objekt enthält Features als Data Frame, welche wiederum Data
Frames enthält.

\begin{Shaded}
\begin{Highlighting}[]
\NormalTok{features }\OtherTok{\textless{}{-}}\NormalTok{ past\_hours\_js}\SpecialCharTok{$}\NormalTok{features}
\NormalTok{features}
\end{Highlighting}
\end{Shaded}

\begin{verbatim}
##      type properties.mag             properties.place properties.time
## 1 Feature           1.92   5 km SSE of Pāhala, Hawaii    1.698512e+12
## 2 Feature           1.90          Southeastern Alaska    1.698511e+12
## 3 Feature           0.97 4 km NNW of Lake Henshaw, CA    1.698511e+12
## 4 Feature           2.20              Southern Alaska    1.698511e+12
## 5 Feature           1.20               Central Alaska    1.698509e+12
##   properties.updated properties.tz
## 1       1.698512e+12            NA
## 2       1.698511e+12            NA
## 3       1.698511e+12            NA
## 4       1.698511e+12            NA
## 5       1.698509e+12            NA
##                                                   properties.url
## 1   https://earthquake.usgs.gov/earthquakes/eventpage/hv73629607
## 2 https://earthquake.usgs.gov/earthquakes/eventpage/ak023du2iu30
## 3   https://earthquake.usgs.gov/earthquakes/eventpage/ci40590336
## 4 https://earthquake.usgs.gov/earthquakes/eventpage/ak023du2gxob
## 5 https://earthquake.usgs.gov/earthquakes/eventpage/ak023du2c8ey
##                                                               properties.detail
## 1   https://earthquake.usgs.gov/earthquakes/feed/v1.0/detail/hv73629607.geojson
## 2 https://earthquake.usgs.gov/earthquakes/feed/v1.0/detail/ak023du2iu30.geojson
## 3   https://earthquake.usgs.gov/earthquakes/feed/v1.0/detail/ci40590336.geojson
## 4 https://earthquake.usgs.gov/earthquakes/feed/v1.0/detail/ak023du2gxob.geojson
## 5 https://earthquake.usgs.gov/earthquakes/feed/v1.0/detail/ak023du2c8ey.geojson
##   properties.felt properties.cdi properties.mmi properties.alert
## 1              NA             NA             NA               NA
## 2              NA             NA             NA               NA
## 3              NA             NA             NA               NA
## 4              NA             NA             NA               NA
## 5              NA             NA             NA               NA
##   properties.status properties.tsunami properties.sig properties.net
## 1         automatic                  0             57             hv
## 2         automatic                  0             56             ak
## 3         automatic                  0             14             ci
## 4         automatic                  0             74             ak
## 5         automatic                  0             22             ak
##   properties.code properties.ids properties.sources
## 1        73629607   ,hv73629607,               ,hv,
## 2      023du2iu30 ,ak023du2iu30,               ,ak,
## 3        40590336   ,ci40590336,               ,ci,
## 4      023du2gxob ,ak023du2gxob,               ,ak,
## 5      023du2c8ey ,ak023du2c8ey,               ,ak,
##                                 properties.types properties.nst properties.dmin
## 1                            ,origin,phase-data,             34              NA
## 2                            ,origin,phase-data,             NA              NA
## 3 ,nearby-cities,origin,phase-data,scitech-link,             28          0.0384
## 4                            ,origin,phase-data,             NA              NA
## 5                            ,origin,phase-data,             NA              NA
##   properties.rms properties.gap properties.magType properties.type
## 1           0.15            172                 md      earthquake
## 2           0.87             NA                 ml      earthquake
## 3           0.21             65                 ml      earthquake
## 4           0.72             NA                 ml      earthquake
## 5           0.72             NA                 ml      earthquake
##                       properties.title geometry.type
## 1   M 1.9 - 5 km SSE of Pāhala, Hawaii         Point
## 2          M 1.9 - Southeastern Alaska         Point
## 3 M 1.0 - 4 km NNW of Lake Henshaw, CA         Point
## 4              M 2.2 - Southern Alaska         Point
## 5               M 1.2 - Central Alaska         Point
##             geometry.coordinates           id
## 1 -155.46249, 19.16017, 33.75000   hv73629607
## 2     -136.5545, 58.9818, 0.0000 ak023du2iu30
## 3 -116.78167, 33.26933, 11.68000   ci40590336
## 4    -146.8878, 61.3049, 17.6000 ak023du2gxob
## 5     -144.0529, 62.8949, 3.6000 ak023du2c8ey
\end{verbatim}

Das Data Frame geometry enthält die Koordinaten. Diese werden in
einzelne Spalten aufgeteilt. Die Werte werde in einer Matrix
gespeichert.

\begin{Shaded}
\begin{Highlighting}[]
\NormalTok{geometry }\OtherTok{\textless{}{-}}\NormalTok{ features}\SpecialCharTok{$}\NormalTok{geometry}
\NormalTok{coordinates }\OtherTok{\textless{}{-}}\NormalTok{ geometry}\SpecialCharTok{$}\NormalTok{coordinates}
\NormalTok{coordinates }\OtherTok{\textless{}{-}} \FunctionTok{do.call}\NormalTok{(rbind, coordinates)}
\FunctionTok{colnames}\NormalTok{(coordinates) }\OtherTok{\textless{}{-}} \FunctionTok{c}\NormalTok{(}\StringTok{"lon"}\NormalTok{, }\StringTok{"lat"}\NormalTok{, }\StringTok{"depth"}\NormalTok{)}
\end{Highlighting}
\end{Shaded}

\hypertarget{daten-zu-einem-dataframe-zusammenfuxfchren}{%
\subsubsection{Daten zu einem Dataframe
zusammenführen}\label{daten-zu-einem-dataframe-zusammenfuxfchren}}

Nun kann das weitere Data Frame properties mit den Koordinaten-Matrix
verbunden und in einem Dataframe für die Exploration gespeichert werden.

\begin{Shaded}
\begin{Highlighting}[]
\NormalTok{properties\_df }\OtherTok{\textless{}{-}} \FunctionTok{cbind}\NormalTok{(features}\SpecialCharTok{$}\NormalTok{properties, coordinates)}
\FunctionTok{head}\NormalTok{(properties\_df)}
\end{Highlighting}
\end{Shaded}

\begin{verbatim}
##    mag                        place         time      updated tz
## 1 1.92   5 km SSE of Pāhala, Hawaii 1.698512e+12 1.698512e+12 NA
## 2 1.90          Southeastern Alaska 1.698511e+12 1.698511e+12 NA
## 3 0.97 4 km NNW of Lake Henshaw, CA 1.698511e+12 1.698511e+12 NA
## 4 2.20              Southern Alaska 1.698511e+12 1.698511e+12 NA
## 5 1.20               Central Alaska 1.698509e+12 1.698509e+12 NA
##                                                              url
## 1   https://earthquake.usgs.gov/earthquakes/eventpage/hv73629607
## 2 https://earthquake.usgs.gov/earthquakes/eventpage/ak023du2iu30
## 3   https://earthquake.usgs.gov/earthquakes/eventpage/ci40590336
## 4 https://earthquake.usgs.gov/earthquakes/eventpage/ak023du2gxob
## 5 https://earthquake.usgs.gov/earthquakes/eventpage/ak023du2c8ey
##                                                                          detail
## 1   https://earthquake.usgs.gov/earthquakes/feed/v1.0/detail/hv73629607.geojson
## 2 https://earthquake.usgs.gov/earthquakes/feed/v1.0/detail/ak023du2iu30.geojson
## 3   https://earthquake.usgs.gov/earthquakes/feed/v1.0/detail/ci40590336.geojson
## 4 https://earthquake.usgs.gov/earthquakes/feed/v1.0/detail/ak023du2gxob.geojson
## 5 https://earthquake.usgs.gov/earthquakes/feed/v1.0/detail/ak023du2c8ey.geojson
##   felt cdi mmi alert    status tsunami sig net       code            ids
## 1   NA  NA  NA    NA automatic       0  57  hv   73629607   ,hv73629607,
## 2   NA  NA  NA    NA automatic       0  56  ak 023du2iu30 ,ak023du2iu30,
## 3   NA  NA  NA    NA automatic       0  14  ci   40590336   ,ci40590336,
## 4   NA  NA  NA    NA automatic       0  74  ak 023du2gxob ,ak023du2gxob,
## 5   NA  NA  NA    NA automatic       0  22  ak 023du2c8ey ,ak023du2c8ey,
##   sources                                          types nst   dmin  rms gap
## 1    ,hv,                            ,origin,phase-data,  34     NA 0.15 172
## 2    ,ak,                            ,origin,phase-data,  NA     NA 0.87  NA
## 3    ,ci, ,nearby-cities,origin,phase-data,scitech-link,  28 0.0384 0.21  65
## 4    ,ak,                            ,origin,phase-data,  NA     NA 0.72  NA
## 5    ,ak,                            ,origin,phase-data,  NA     NA 0.72  NA
##   magType       type                                title       lon      lat
## 1      md earthquake   M 1.9 - 5 km SSE of Pāhala, Hawaii -155.4625 19.16017
## 2      ml earthquake          M 1.9 - Southeastern Alaska -136.5545 58.98180
## 3      ml earthquake M 1.0 - 4 km NNW of Lake Henshaw, CA -116.7817 33.26933
## 4      ml earthquake              M 2.2 - Southern Alaska -146.8878 61.30490
## 5      ml earthquake               M 1.2 - Central Alaska -144.0529 62.89490
##   depth
## 1 33.75
## 2  0.00
## 3 11.68
## 4 17.60
## 5  3.60
\end{verbatim}

\hypertarget{entfernen-von-spalten}{%
\subsubsection{Entfernen von Spalten}\label{entfernen-von-spalten}}

Für unsere Problemstellung benötigen wir nicht alle gelieferten Spalten.
Um entscheiden zu können, welche Spalten für die Weiterarbeit
weggelassen werden, schauen wir uns als erstes den Datenbeschrieb an.

\emph{Datenbeschrieb}

mag: Werte von -1 bis 10 Die angegebene Stärke ist diejenige, die der
U.S. Geological Survey für dieses Erdbeben als offiziell ansieht, und
war die beste verfügbare Schätzung der Größe des Erdbebens zum Zeitpunkt
der Erstellung dieser Seite. Andere Magnituden, die mit den von hier aus
verlinkten Webseiten verbunden sind, wurden zu verschiedenen Zeitpunkten
nach dem Erdbeben mit verschiedenen Arten von seismischen Daten
ermittelt. Obwohl es sich dabei um legitime Schätzungen der Stärke
handelt, betrachtet der U.S. Geological Survey sie nicht als die
bevorzugte ``offizielle'' Stärke für das Ereignis.

Die Erdbebenstärke ist ein Maß für die Größe eines Erdbebens an seinem
Ursprung. Es handelt sich um ein logarithmisches Maß. Bei gleichem
Abstand vom Erdbeben ist die Amplitude der seismischen Wellen, aus denen
die Magnitude bestimmt wird, bei einem Erdbeben der Magnitude 5 etwa
zehnmal so groß wie bei einem Erdbeben der Magnitude 4. Die
Gesamtenergiemenge, die durch das Erdbeben freigesetzt wird, steigt in
der Regel um einen größeren Faktor: Bei vielen gebräuchlichen
Magnitudenarten steigt die Gesamtenergie eines durchschnittlichen
Erdbebens um einen Faktor von etwa 32 für jede Einheit, um die die
Magnitude zunimmt.

Es gibt verschiedene Möglichkeiten, die Magnitude aus Seismogrammen zu
berechnen. Die verschiedenen Methoden sind für unterschiedliche
Erdbebengrößen und unterschiedliche Entfernungen zwischen der
Erdbebenquelle und der Aufzeichnungsstation geeignet. Die verschiedenen
Magnituden-Typen sind in der Regel so definiert, dass die
Magnitudenwerte für Erdbeben in einem mittleren Bereich der
aufgezeichneten Erdbebengrößen innerhalb weniger Zehntel einer
Magnitudeneinheit übereinstimmen, aber die verschiedenen
Magnituden-Typen können Werte aufweisen, die sich für sehr große und
sehr kleine Erdbeben sowie für einige spezifische Klassen seismischer
Quellen um mehr als eine Magnitudeneinheit unterscheiden. Dies ist
darauf zurückzuführen, dass Erdbeben in der Regel komplexe Ereignisse
sind, bei denen im Zuge des Verwerfungs- oder Bruchprozesses Energie in
einem breiten Frequenzbereich und in unterschiedlicher Stärke
freigesetzt wird. Die verschiedenen Arten von Magnituden messen
unterschiedliche Aspekte der seismischen Strahlung (z. B.
niederfrequente Energie gegenüber hochfrequenter Energie). Die Beziehung
zwischen den Werten der verschiedenen Magnitudenarten, die einem
bestimmten seismischen Ereignis zugeordnet werden, kann dem Seismologen
ein besseres Verständnis der Prozesse im Brennpunkt des seismischen
Ereignisses ermöglichen. Die verschiedenen Magnituden-Typen sind nicht
alle gleichzeitig für ein bestimmtes Erdbeben verfügbar.

Manchmal werden vorläufige Größenordnungen auf der Grundlage
unvollständiger, aber schnell verfügbarer Daten geschätzt und gemeldet.
Beispielsweise berechnen die Tsunami-Warnzentren eine vorläufige
Magnitude und den Ort eines Ereignisses, sobald genügend Daten für eine
Schätzung vorliegen. In diesem Fall ist die Zeit von entscheidender
Bedeutung, um eine Warnung auszusenden, wenn durch das Ereignis
wahrscheinlich Tsunami-Wellen ausgelöst werden. Solche vorläufigen
Magnituden werden durch verbesserte Schätzungen der Magnitude ersetzt,
sobald mehr Daten zur Verfügung stehen.

Bei großen Erdbeben der heutigen Zeit ist die Magnitude, die letztlich
als bevorzugte Magnitude für die Berichterstattung an die Öffentlichkeit
ausgewählt wird, in der Regel eine Momentmagnitude, die auf dem skalaren
seismischen Moment eines Erdbebens basiert, das durch Berechnung des
seismischen Momenttensors bestimmt wird, der den Charakter der vom
Erdbeben erzeugten seismischen Wellen am besten wiedergibt. Das skalare
seismische Moment, ein Parameter des seismischen Momententensors, kann
auch über das multiplikative Produkt Steifigkeit des gestörten Gesteins
x Bruchfläche x durchschnittliche Verwerfungsverschiebung während des
Erdbebens geschätzt werden.

rms: root-mean-square (RMS) Der quadratische Mittelwert (RMS) der
Reisezeitresiduen in Sekunden unter Verwendung aller Gewichte ist ein
Maß für die Übereinstimmung zwischen den beobachteten Ankunftszeiten und
den vorhergesagten Ankunftszeiten an diesem Ort. Kleinere Werte deuten
auf eine bessere Anpassung der Daten hin. Der Wert ist abhängig von der
Genauigkeit des verwendeten Geschwindigkeitsmodells zur Berechnung der
Erdbebenposition, der Gewichtung der Qualität der Ankunftszeitdaten und
dem angewandten Verfahren zur Lokalisierung des Erdbebens. Typische
Daten bewegen sich zwischen 0.13 ud 1.39

nst: Die Anzahl der Seismostationen, die zur Bestimmung des Standorts
eines Erdbebens verwendet werden.

dmin: Horizontale Entfernung in Grad vom Epizentrum zur nächstgelegenen
Station. 1 Grad entspricht ungefähr 111,2 Kilometern. Im Allgemeinen
gilt: Je geringer dieser Wert ist, desto zuverlässiger ist die
berechnete Tiefe des Erdbebens. Typische Werte bewegen sich zwischen 0.4
und 7.1

gap: Der größte azimutale Abstand zwischen benachbarten Stationen in
Grad beeinflusst die Zuverlässigkeit der berechneten horizontalen
Position des Erdbebens. Eine geringere Distanz bedeutet im Allgemeinen
höhere Genauigkeit. Größere azimutale Lücken als 180 Grad zeigen in der
Regel große Unsicherheiten in Bezug auf die Lage und Tiefe des
Erdbebens. Typischer Wertebereich: 0-180

magType: Die Methode oder der Algorithmus, der zur Berechnung der
bevorzugten Größenordnung des Ereignisses verwendet wird. Werte: ``Md'',
``Ml'', ``Ms'', ``Mw'', ``Me'', ``Mi'', ``Mb'', ``MLg'' Hier werden
diese Typen genauer beschrieben:
\url{https://www.usgs.gov/programs/earthquake-hazards/magnitude-types}

tz: Zeitzonenabweichung von der UTC in Minuten am Epizentrum des
Ereignisses.

net: Die ID eines Datenlieferanten. Kennzeichnet das Netz, das als
bevorzugte Informationsquelle für dieses Ereignis gilt. Werte: ak, at,
ci, hv, ld, mb, nc, nm, nn, pr, pt, se, us, uu, uw

sig: Eine Bewertungszahl, die die Bedeutung eines Ereignisses anzeigt.
Je höher die Zahl, desto bedeutender das Ereignis. Dieser Wert wird
anhand einer Reihe von Faktoren wie Ausmaß, maximaler MMI, berichteten
Empfindungen und geschätzten Auswirkungen ermittelt.

ids: Eine durch Kommata getrennte Liste von Ereignis-IDs, die mit einem
Ereignis verknüpft sind.

code: Ein von der entsprechenden Quelle für das Ereignis zugewiesener -
und eindeutiger - Identifizierungscode.

sources: Eine durch Kommata getrennte Liste von Netzwerkteilnehmern.

tsunami: Dieses Flag wird bei großen Ereignissen in ozeanischen Regionen
auf ``1'' gesetzt und ansonsten auf ``0''. Das Vorhandensein oder der
Wert dieses Flags sagt nichts darüber aus, ob tatsächlich ein Tsunami
aufgetreten ist oder auftreten wird. Wenn das Flag den Wert ``1'' hat,
enthält das Ereignis einen Link zur NOAA Tsunami-Website für
Tsunami-Informationen.

felt: Die Gesamtzahl der an das DYFI-System übermittelten Spürmeldungen.
Did You Feel It? (DYFI) sammelt Informationen von Menschen, die ein
Erdbeben gespürt haben, und erstellt Karten, die zeigen, was die
Menschen erlebt haben und wie groß die Schäden sind. Werte zwischen 44
und 843

cdi: Die höchste gemeldete Intensität für das Ereignis. Berechnet von
DYFI. Werte von 0-10

mmi: Die maximale geschätzte instrumentelle Intensität für das Ereignis.
Berechnet von ShakeMap. ShakeMap ist ein Produkt des USGS Earthquake
Hazards Program in Verbindung mit den regionalen seismischen Netzen.
ShakeMaps liefern nahezu in Echtzeit Karten der Bodenbewegungen und der
Erschütterungsintensität nach schweren Erdbeben. Diese Karten werden von
öffentlichen und privaten Organisationen auf Bundes-, Landes- und
Kommunalebene für die Reaktion und den Wiederaufbau nach einem Erdbeben,
für öffentliche und wissenschaftliche Informationen sowie für
Bereitschaftsübungen und Katastrophenplanung verwendet. Werte von 0-10

alert: Die Alarmstufe der PAGER-Skala für Erdbebenauswirkungen. Mögliche
Werte: ``green'', ``yellow'', ``orange'', ``red''.

status: Zeigt an, ob das Ereignis von einem Menschen überprüft wurde.
Der Status ist entweder automatisch oder geprüft. Automatische
Ereignisse werden direkt von automatischen Verarbeitungssystemen gebucht
und wurden nicht von einem Menschen überprüft oder geändert. Überprüfte
Ereignisse wurden von einem Menschen geprüft. Der Grad der Überprüfung
kann von einer schnellen Gültigkeitsprüfung bis hin zu einer
sorgfältigen Neuanalyse des Ereignisses reichen. Mögliche Werte:
``automatic'', ``reviewed'', ``deleted''

type: Art des seismischen Ereignisses.''Erdbeben'', ``Steinbruch''
Typische Werte: ``earthquake'', ``quarry''

lon: Längengrad im WGS84-System angegeben. Die Werte bewegen sich
zwischen -180.0 und 180.0.

lat: Breitengrad im WGS84-System angegeben. Die Werte bewegen sich
zwischen -90.0 und 90.0.

depth: Tiefe des Ereignisses von der Erdoberfläche gemessen in
Kilometer. Werte von 0 bis 1000.

Aus diesem Datenbeschrieb geht hervor, dass wir die Spalten mag, place,
time, status, updated, type, longitude, latitude, depth, tsunami, sig,
code behalten werden und alle anderen Spalten aus dem Data Frame
entfernen.

\begin{Shaded}
\begin{Highlighting}[]
\NormalTok{df }\OtherTok{\textless{}{-}}\NormalTok{ properties\_df }\SpecialCharTok{\%\textgreater{}\%} \FunctionTok{select}\NormalTok{(mag, place, time, status, updated, type, lon, lat, depth, tsunami, sig, code)}
\end{Highlighting}
\end{Shaded}

\hypertarget{zeitstempel-konvertieren}{%
\subsubsection{Zeitstempel
konvertieren}\label{zeitstempel-konvertieren}}

Wie wir bereits gesehen haben, müssen die Spalten mit den Zeitstempeln
(time und updated) noch konvertiert werden. Der Zeitstempel wird in
Millisekunden seit der Epoche (1970-01-01T00:00:00.000Z), angegeben,
ohne Berücksichtigung von Schaltsekunden.

Dazu wird für die Berechnung das Basisdatum addiert mit dem vorhandenen
Wert, welcher wiederum durch 1000 geteilt wird.

\begin{Shaded}
\begin{Highlighting}[]
\CommentTok{\# properties\_df \textless{}{-} properties\_df \%\textgreater{}\% }
\NormalTok{basedate }\OtherTok{\textless{}{-}} \FunctionTok{as.POSIXct}\NormalTok{(}\StringTok{"1970{-}01{-}01 00:00:00"}\NormalTok{, }\AttributeTok{tz =} \StringTok{"UTC"}\NormalTok{)}

\NormalTok{df }\OtherTok{\textless{}{-}}\NormalTok{ df }\SpecialCharTok{\%\textgreater{}\%} 
  \FunctionTok{mutate}\NormalTok{(}\AttributeTok{time =}\NormalTok{ basedate }\SpecialCharTok{+}\NormalTok{ (time}\SpecialCharTok{/} \DecValTok{1000}\NormalTok{)) }\SpecialCharTok{\%\textgreater{}\%} 
  \FunctionTok{mutate}\NormalTok{(}\AttributeTok{updated =}\NormalTok{ basedate }\SpecialCharTok{+}\NormalTok{ (updated}\SpecialCharTok{/}\DecValTok{1000}\NormalTok{))}
\end{Highlighting}
\end{Shaded}

Falls in den Spalten mag, place, time, type, longitude, latitude, depth,
code fehlende Werte vorhanden sind, ist eine Verwendung dieser
Observationen nicht sinnvoll. Deshalb werden diese Observationen aus dem
Datensatz entfernt.

\begin{Shaded}
\begin{Highlighting}[]
\NormalTok{df }\OtherTok{\textless{}{-}}\NormalTok{ df }\SpecialCharTok{\%\textgreater{}\%} 
  \FunctionTok{drop\_na}\NormalTok{(mag, place, time, type, lon, lat, depth, code)}
\end{Highlighting}
\end{Shaded}

\hypertarget{uxfcberpruxfcfung-der-zuluxe4ssigen-werte}{%
\subsubsection{Überprüfung der zulässigen
Werte}\label{uxfcberpruxfcfung-der-zuluxe4ssigen-werte}}

Wie im Datenbeschrieb zu entnehmen, dürfen sich die Werte für den
Längengrad im Bereich von -180 bis 180, für den Breitengrad von -90 bis
90 und für die Tiefe von 0 bis 1000 bewegen. Dies wird anhand einer
Abfrage überprüft. Dazu wird eine zusätzliche Spalte für die jeweilige
Variable erstellt und mit einem booleanschen Operator versehen. Der Wert
True steht jeweils für einen Wert, der sich im zulässigen Bereich
befindet.

\begin{Shaded}
\begin{Highlighting}[]
\NormalTok{df }\OtherTok{\textless{}{-}}\NormalTok{ df }\SpecialCharTok{\%\textgreater{}\%}
  \FunctionTok{mutate}\NormalTok{(}\AttributeTok{lon\_inrange =} \FunctionTok{between}\NormalTok{(lon, }\SpecialCharTok{{-}}\DecValTok{180}\NormalTok{, }\DecValTok{180}\NormalTok{),}
         \AttributeTok{lat\_inrange =} \FunctionTok{between}\NormalTok{(lat, }\SpecialCharTok{{-}}\DecValTok{90}\NormalTok{, }\DecValTok{90}\NormalTok{),}
         \AttributeTok{depth\_inrange =} \FunctionTok{between}\NormalTok{(depth, }\DecValTok{0}\NormalTok{, }\DecValTok{1000}\NormalTok{),}
         \AttributeTok{mag\_inrange =} \FunctionTok{between}\NormalTok{(mag, }\SpecialCharTok{{-}}\DecValTok{1}\NormalTok{, }\DecValTok{10}\NormalTok{))}
\end{Highlighting}
\end{Shaded}

Werte, welche sich nicht im zulässigen Bereich befinden, in einem neuen
Data Frame anzeigen lassen.

\begin{Shaded}
\begin{Highlighting}[]
\NormalTok{not\_inrange }\OtherTok{\textless{}{-}}\NormalTok{ df }\SpecialCharTok{\%\textgreater{}\%} 
  \FunctionTok{filter}\NormalTok{(lon\_inrange }\SpecialCharTok{==} \ConstantTok{FALSE}\NormalTok{,}
\NormalTok{         lat\_inrange }\SpecialCharTok{==} \ConstantTok{FALSE}\NormalTok{, }
\NormalTok{         depth\_inrange }\SpecialCharTok{==} \ConstantTok{FALSE}\NormalTok{,}
\NormalTok{         mag\_inrange }\SpecialCharTok{==} \ConstantTok{FALSE}\NormalTok{)}
\NormalTok{not\_inrange}
\end{Highlighting}
\end{Shaded}

\begin{verbatim}
##  [1] mag           place         time          status        updated      
##  [6] type          lon           lat           depth         tsunami      
## [11] sig           code          lon_inrange   lat_inrange   depth_inrange
## [16] mag_inrange  
## <0 Zeilen> (oder row.names mit Länge 0)
\end{verbatim}

Falls sich Werte nicht im zulässigen Bereich befinden, werden diese
Observationen entfernt.

\begin{Shaded}
\begin{Highlighting}[]
\NormalTok{df }\OtherTok{\textless{}{-}}\NormalTok{ df }\SpecialCharTok{\%\textgreater{}\%} 
  \FunctionTok{filter}\NormalTok{(lon\_inrange }\SpecialCharTok{==} \ConstantTok{TRUE}\NormalTok{,}
\NormalTok{         lat\_inrange }\SpecialCharTok{==} \ConstantTok{TRUE}\NormalTok{, }
\NormalTok{         depth\_inrange }\SpecialCharTok{==} \ConstantTok{TRUE}\NormalTok{,}
\NormalTok{         mag\_inrange }\SpecialCharTok{==} \ConstantTok{TRUE}\NormalTok{)}
\NormalTok{df}
\end{Highlighting}
\end{Shaded}

\begin{verbatim}
##    mag                        place                time    status
## 1 1.92   5 km SSE of Pāhala, Hawaii 2023-10-28 16:46:32 automatic
## 2 1.90          Southeastern Alaska 2023-10-28 16:39:36 automatic
## 3 0.97 4 km NNW of Lake Henshaw, CA 2023-10-28 16:34:33 automatic
## 4 2.20              Southern Alaska 2023-10-28 16:30:50 automatic
## 5 1.20               Central Alaska 2023-10-28 16:08:56 automatic
##               updated       type       lon      lat depth tsunami sig
## 1 2023-10-28 16:49:41 earthquake -155.4625 19.16017 33.75       0  57
## 2 2023-10-28 16:41:36 earthquake -136.5545 58.98180  0.00       0  56
## 3 2023-10-28 16:38:11 earthquake -116.7817 33.26933 11.68       0  14
## 4 2023-10-28 16:32:39 earthquake -146.8878 61.30490 17.60       0  74
## 5 2023-10-28 16:11:15 earthquake -144.0529 62.89490  3.60       0  22
##         code lon_inrange lat_inrange depth_inrange mag_inrange
## 1   73629607        TRUE        TRUE          TRUE        TRUE
## 2 023du2iu30        TRUE        TRUE          TRUE        TRUE
## 3   40590336        TRUE        TRUE          TRUE        TRUE
## 4 023du2gxob        TRUE        TRUE          TRUE        TRUE
## 5 023du2c8ey        TRUE        TRUE          TRUE        TRUE
\end{verbatim}

\hypertarget{explorative-analyse}{%
\subsubsection{Explorative Analyse}\label{explorative-analyse}}

Verteilung der Erdbebenstärken (Magnituden) im Datenset. Dazu wird eine
Visualisierung erstellt.

\begin{Shaded}
\begin{Highlighting}[]
\FunctionTok{ggplot}\NormalTok{(df, }\FunctionTok{aes}\NormalTok{(}\AttributeTok{y =}\NormalTok{ mag)) }\SpecialCharTok{+}
  \FunctionTok{geom\_boxplot}\NormalTok{(}\AttributeTok{fill=}\StringTok{"blue"}\NormalTok{, }\AttributeTok{color=}\StringTok{"black"}\NormalTok{, }\AttributeTok{alpha=}\FloatTok{0.7}\NormalTok{) }\SpecialCharTok{+}
  \FunctionTok{theme\_minimal}\NormalTok{() }\SpecialCharTok{+}
  \FunctionTok{labs}\NormalTok{(}\AttributeTok{title=}\StringTok{"Verteilung der Magnituden"}\NormalTok{, }\AttributeTok{y=}\StringTok{"Erdbebenstärke"}\NormalTok{) }
\end{Highlighting}
\end{Shaded}

\includegraphics{MC_Erdbeben_files/figure-latex/unnamed-chunk-14-1.pdf}
Nun wollen wir schauen, wei die Tiefen der Epizentren verteilt sind

\begin{Shaded}
\begin{Highlighting}[]
\FunctionTok{ggplot}\NormalTok{(df, }\FunctionTok{aes}\NormalTok{(}\AttributeTok{y =}\NormalTok{ depth)) }\SpecialCharTok{+}
  \FunctionTok{geom\_boxplot}\NormalTok{(}\AttributeTok{fill=}\StringTok{"blue"}\NormalTok{, }\AttributeTok{color=}\StringTok{"black"}\NormalTok{, }\AttributeTok{alpha=}\FloatTok{0.7}\NormalTok{) }\SpecialCharTok{+}
  \FunctionTok{theme\_minimal}\NormalTok{() }\SpecialCharTok{+}
  \FunctionTok{labs}\NormalTok{(}\AttributeTok{title=}\StringTok{"Verteilung der Tiefen der Epizentren"}\NormalTok{, }\AttributeTok{y=}\StringTok{"Tiefe in km"}\NormalTok{, }\AttributeTok{y=}\StringTok{""}\NormalTok{) }
\end{Highlighting}
\end{Shaded}

\includegraphics{MC_Erdbeben_files/figure-latex/unnamed-chunk-15-1.pdf}
Sind Tsunami-Warngungen ausgegeben worden?

\begin{Shaded}
\begin{Highlighting}[]
\NormalTok{df }\SpecialCharTok{\%\textgreater{}\%} \FunctionTok{filter}\NormalTok{(tsunami }\SpecialCharTok{!=} \DecValTok{0}\NormalTok{)}
\end{Highlighting}
\end{Shaded}

\begin{verbatim}
##  [1] mag           place         time          status        updated      
##  [6] type          lon           lat           depth         tsunami      
## [11] sig           code          lon_inrange   lat_inrange   depth_inrange
## [16] mag_inrange  
## <0 Zeilen> (oder row.names mit Länge 0)
\end{verbatim}

\hypertarget{visualisierung-auf-der-weltkarte}{%
\subsubsection{Visualisierung auf der
Weltkarte}\label{visualisierung-auf-der-weltkarte}}

Für die Visualisierung auf einer Weltkarte kann die Bibliothek
``leaflet'' verwendet werden.

\begin{Shaded}
\begin{Highlighting}[]
\NormalTok{map }\OtherTok{\textless{}{-}} \FunctionTok{leaflet}\NormalTok{(df) }\SpecialCharTok{\%\textgreater{}\%} 
  \FunctionTok{addTiles}\NormalTok{() }\SpecialCharTok{\%\textgreater{}\%} 
  \FunctionTok{addCircleMarkers}\NormalTok{(}
    \AttributeTok{lat =} \SpecialCharTok{\textasciitilde{}}\NormalTok{lat,}
    \AttributeTok{lng =} \SpecialCharTok{\textasciitilde{}}\NormalTok{lon,}
    \AttributeTok{radius =} \SpecialCharTok{\textasciitilde{}}\NormalTok{mag }\SpecialCharTok{*}\DecValTok{2}\NormalTok{,  }\CommentTok{\# Größe der Punkte nach Magnitude}
    \AttributeTok{fillColor =} \SpecialCharTok{\textasciitilde{}}\FunctionTok{colorQuantile}\NormalTok{(}\StringTok{"YlOrRd"}\NormalTok{, mag)(mag),  }\CommentTok{\# Farbschema}
    \AttributeTok{color =} \StringTok{"\#000000"}\NormalTok{,  }\CommentTok{\# Umrandungsfarbe}
    \AttributeTok{weight =} \DecValTok{1}\NormalTok{,  }\CommentTok{\# Umrandungsdicke}
    \AttributeTok{opacity =} \DecValTok{1}\NormalTok{,  }\CommentTok{\# Umrandungstransparenz}
    \AttributeTok{fillOpacity =} \FloatTok{0.7}\NormalTok{,  }\CommentTok{\# Fülltransparenz}
    \AttributeTok{popup =} \SpecialCharTok{\textasciitilde{}}\FunctionTok{paste}\NormalTok{(}\StringTok{"Ort:"}\NormalTok{, place, }\StringTok{"\textless{}br\textgreater{}Magnitude:"}\NormalTok{, }\FunctionTok{round}\NormalTok{(mag, }\DecValTok{4}\NormalTok{), }\StringTok{"\textless{}br\textgreater{}Tiefe:"}\NormalTok{, }\FunctionTok{round}\NormalTok{(df}\SpecialCharTok{$}\NormalTok{depth, }\DecValTok{2}\NormalTok{))  }\CommentTok{\# Popup{-}Info}
\NormalTok{  )}
\FunctionTok{print}\NormalTok{(map)}
\end{Highlighting}
\end{Shaded}

Für einen Kartenausschnitt, in dem alle vorhandenen Erdbeben ersichtlich
sind, kann die bbox verwendet werden. Die Daten werden im JSON-Objekt
mitgeliefert. Für die Verwendung in einem Leaflet werden diese
Informationen in ein Data Frame gespeichert.

\begin{Shaded}
\begin{Highlighting}[]
\NormalTok{bbox }\OtherTok{\textless{}{-}}\NormalTok{ past\_hours\_js}\SpecialCharTok{$}\NormalTok{bbox}
\NormalTok{bbox }\OtherTok{\textless{}{-}} \FunctionTok{data.frame}\NormalTok{(bbox, }\AttributeTok{coord =} \FunctionTok{c}\NormalTok{(}\StringTok{"lon\_min"}\NormalTok{, }\StringTok{"lat\_min"}\NormalTok{, }\StringTok{"depth\_min"}\NormalTok{, }\StringTok{"lon\_max"}\NormalTok{, }\StringTok{"lat\_max"}\NormalTok{, }\StringTok{"depth\_max"}\NormalTok{))}
\NormalTok{bbox}
\end{Highlighting}
\end{Shaded}

\begin{verbatim}
##         bbox     coord
## 1 -155.46249   lon_min
## 2   19.16017   lat_min
## 3    0.00000 depth_min
## 4 -116.78167   lon_max
## 5   62.89490   lat_max
## 6   33.75000 depth_max
\end{verbatim}

\end{document}
